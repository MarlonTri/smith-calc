%%% template.tex
%%% This is a template for making up an AMS-LaTeX file
%%% Version of February 12, 2011
%%%---------------------------------------------------------
%%% The following command chooses the default 10 point type.
%%% To choose 12 point, change it to
%%% \documentclass[12pt]{amsart}
\documentclass{amsart}

%%% The following command loads the amsrefs package, which will be
%%% used to create the bibliography:
\usepackage[lite]{amsrefs}

%%% The following command defines the standard names for all of the
%%% special symbols in the AMSfonts package, listed in
%%% http://www.ctan.org/tex-archive/info/symbols/math/symbols.pdf
\usepackage{amssymb}

%%% The following commands allow you to use \Xy-pic to draw
%%% commutative diagrams.  (You can omit the second line if you want
%%% the default style of the nodes to be \textstyle.)
\usepackage[all,cmtip]{xy}
\let\objectstyle=\displaystyle

%%% If you'll be importing any graphics, uncomment the following
%%% line.  (Note: The spelling is correct; the package graphicx.sty is
%%% the updated version of the older graphics.sty.)
% \usepackage{graphicx}



%%% This part of the file (after the \documentclass command,
%%% but before the \begin{document}) is called the ``preamble''.
%%% This is where we put our macro definitions.

%%% Comment out (or delete) any of these that you don't want to use.
\newcommand{\tensor}{\otimes}
\newcommand{\homotopic}{\simeq}
\newcommand{\homeq}{\cong}
\newcommand{\iso}{\approx}

\DeclareMathOperator{\ho}{Ho}
\DeclareMathOperator*{\colim}{colim}

\newcommand{\R}{\mathbb{R}}
\newcommand{\C}{\mathbb{C}}
\newcommand{\Z}{\mathbb{Z}}

\newcommand{\M}{\mathcal{M}}
\newcommand{\W}{\mathcal{W}}

\newcommand{\itilde}{\tilde{\imath}}
\newcommand{\jtilde}{\tilde{\jmath}}
\newcommand{\ihat}{\hat{\imath}}
\newcommand{\jhat}{\hat{\jmath}}



%%%-------------------------------------------------------------------
%%%-------------------------------------------------------------------
%%% The Theorem environments:
%%%
%%%
%%% The following commands set it up so that:
%%% 
%%% All Theorems, Corollaries, Lemmas, Propositions, Definitions,
%%% Remarks, Examples, Notations, and Terminologies  will be numbered
%%% in a single sequence, and the numbering will be within each
%%% section.  Displayed equations will be numbered in the same
%%% sequence. 
%%% 
%%% 
%%% Theorems, Propositions, Lemmas, and Corollaries will have the most
%%% formal typesetting.
%%% 
%%% Definitions will have the next level of formality.
%%% 
%%% Remarks, Examples, Notations, and Terminologies will be the least
%%% formal.
%%% 
%%% Theorem:
%%% \begin{thm}
%%% 
%%% \end{thm}
%%% 
%%% Corollary:
%%% \begin{cor}
%%% 
%%% \end{cor}
%%% 
%%% Lemma:
%%% \begin{lem}
%%% 
%%% \end{lem}
%%% 
%%% Proposition:
%%% \begin{prop}
%%% 
%%% \end{prop}
%%% 
%%% Definition:
%%% \begin{defn}
%%% 
%%% \end{defn}
%%% 
%%% Remark:
%%% \begin{rem}
%%% 
%%% \end{rem}
%%% 
%%% Example:
%%% \begin{ex}
%%% 
%%% \end{ex}
%%% 
%%% Notation:
%%% \begin{notation}
%%% 
%%% \end{notation}
%%% 
%%% Terminology:
%%% \begin{terminology}
%%% 
%%% \end{terminology}
%%% 
%%%       Theorem environments

% The following causes equations to be numbered within sections
\numberwithin{equation}{section}

% We'll use the equation counter for all our theorem environments, so
% that everything will be numbered in the same sequence.

%       Theorem environments

\theoremstyle{plain} %% This is the default, anyway
\newtheorem{thm}[equation]{Theorem}
\newtheorem{cor}[equation]{Corollary}
\newtheorem{lem}[equation]{Lemma}
\newtheorem{prop}[equation]{Proposition}

\theoremstyle{definition}
\newtheorem{defn}[equation]{Definition}

\theoremstyle{remark}
\newtheorem{rem}[equation]{Remark}
\newtheorem{ex}[equation]{Example}
\newtheorem{notation}[equation]{Notation}
\newtheorem{terminology}[equation]{Terminology}

%%%-------------------------------------------------------------------
%%%-------------------------------------------------------------------
%%%-------------------------------------------------------------------
%%%-------------------------------------------------------------------
%%%-------------------------------------------------------------------
%%%-------------------------------------------------------------------
%%%-------------------------------------------------------------------
\begin{document}

%%% In the title, use a double backslash "\\" to show a linebreak:
%%% Use one of the following two forms:
%%% \title{Text of the title}
%%% or
%%% \title[Short form for the running head]{Text of the title}
\title{Largest Smith Number}


%%% If there are multiple authors, they're described one at a time:
%%% First author: \author{} \address{} \curraddr{} \email{} \thanks{}
%%% Second author: \author{} \address{} \curraddr{} \email{} \thanks{}
%%% Third author: \author{} \address{} \curraddr{} \email{} \thanks{}
\author{Marlon Trifunovic}

%%% In the address, show linebreaks with double backslashes:
\address{}

%%% Current address is optional.
% \curraddr{}

%%% Email address is optional.
% \email{}


%%% If there's a second author:
% \author{}
% \address{}
% \curraddr{}
% \email{}


%%% To have the current date inserted, use \date{\today}:
\date{\today}

%%% To include an abstract, uncomment the following two lines and type
%%% the abstract in between them:
% \begin{abstract}
% \end{abstract}


\maketitle

%%% To include a table of contents, uncomment the following line:
 \tableofcontents


%-------------------------------------------------------------------
%-------------------------------------------------------------------
 %Start the body of the paper here!  E.G., maybe use:
 \section{Introduction}
 \section{Best without computing}
 \section{Best by adding coefficients}
 \section{Best by overlapping coefficients}
% \label{sec:intro}

% For a numbered display, use
 %\begin{equation}
  % \label{something}
   %The display goes here
% \end{equation}
% and you can refer to it as% \eqref{something}.

 %For an unnumbered display, use
 %\begin{equation*}
 
 %\end{equation*}

 %To import a graphics file, you must have said
 %\usepackage{graphicx}
% in the preamble (i.e., before the \begin{document}).
% Putting it into a figure environment enables it to float to the
% next page if there isn't enough room for it on the current page.
% The \label command must come after the \caption command.
 %\begin{figure}[h]%
   %\includegraphics{filename}
 %  \caption{Some caption}
  % \label{somelabel}
% \end{figure}















%%% -------------------------------------------------------------------
%%% -------------------------------------------------------------------
%%% This is where we create the bibliography.

\begin{bibdiv}
  \begin{biblist}

%%% The format of bibliography items is as in the following examples:
%%% 
%%% \bib{yellowmonster}{book}{
%%%   author={Bousfield, A.K.},
%%%   author={Kan, D.M.},
%%%   title={Homotopy Limits, Completions and Localizations},
%%%   date={1972},
%%%   series={Lecture Notes in Mathematics},
%%%   volume={304},
%%%   publisher={Springer-Verlag},
%%%   address={Berlin-New York}
%%% }

%%% \bib{HA}{book}{
%%%   author={Quillen, Daniel G.},
%%%   title={Homotopical Algebra},
%%%   series={Lecture Notes in Mathematics},
%%%   volume={43},
%%%   publisher={Springer-Verlag},
%%%   address={Berlin-New York},
%%%   date={1967}
%%% }

%%% \bib{serre:shfs}{article}{
%%%   author={Serre, Jean-Pierre},
%%%   title={Homologie Singuli\`ere des Espaces Fibr\'es.  Applications},
%%%   journal={Ann. of Math. (2)},
%%%   date={1951},
%%%   volume={54},
%%%   pages={425--505}
%%% }





  \end{biblist}
\end{bibdiv}

\end{document}